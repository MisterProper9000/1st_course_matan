\begin{center}
	\section{\LARGE{\bf Непрерывные функции}}
\end{center}
\subsection{Понятие непрерывной функции}
\begin{Definition}
	$f$ - функция, определённая на множестве $E \subset \mathbb{R}, x_0 \in E$. $f$ -- непрерывна в т. $x_0$, если
	\begin{itemize}
		\item $\displaystyle \forall \varepsilon > 0 \; \exists \delta > 0\; \forall x \in E: \; \mid x - x_0 \mid \; < \delta \Rightarrow \; \mid f(x) - f(x_0) \mid \; < \varepsilon$.
		\item $\displaystyle \forall V$ -- окружность $(\cdot) f(x_0) \; \exists U$ -- окружность $(\cdot) x_0$, такая, что $f(U \cap E) \subset V$.
		\item $\displaystyle \forall \{x_n\}_{n=1}^{\infty}, x_n \in E, x_n \xrightarrow[n \rightarrow \infty]{} x_0 \Rightarrow f(x_n) \xrightarrow[n \rightarrow \infty]{} f(x_0)$.
		      Если $x_0$ -- предельная для $E$, то можно определить непрерывность в терминах предела.
		\item $f(x_0) = \lim\limits_{x \rightarrow x_0}f(x), f(x) \xrightarrow[x \rightarrow x_0]{} f(x_0)$
	\end{itemize}
\end{Definition}
Обычно рассматривают функции, определённые в тек. окружности $U_0$ точки $x_0$.
В такой ситуации $f$ непрерывна в т. $x_0$, если $\forall V$-окрестности т. $f(x_0)$. $\exists U$ -- окрестность. $x_0$. $f(U) \subset V$.

\subparagraph{Односторонняя непрерывность}
\begin{Definition}
\begin{itemize}
  \item Пусть $f$ -- функция, определённая по крайней мере на промежутке $(x_0 - \delta_0, x_0]$.
  Функция $f$ называется непрерывной в точке $x_0$ слева, если
  $$f(x_0) = f(x_0 - 0).$$
  \item Функция называется непрерывной в точке $x_0$ справа, если
  $$f(x_0) = f(x_0 + 0).$$
\end{itemize}
\end{Definition}
\begin{Proposition}
Пусть $f$ -- функция, определённая в окрестности точки $x_0$. Для непрерывности функции $f$ в точке $x_0$ необходима и достаточна её непрерывность в этой точке слева и справа.
\end{Proposition}
\subparagraph{Приращение функции}
\begin{Definition}
  Пусть $f$ -- функция, определённая в окрестности точки $x_0$. Определим функцию $\Delta f(x_0)$, полагая
    $$\Delta f(x_0)(h) = f(x_0 + h) - f(x_0)$$
  для тех $h$, для которых $x_0 + h$ лежит в множестве определения функции.
  Функция $\Delta f(x_0)$ называется приращением функции $f$ в точке $x_0$. Приращение определяется в некоторой окрестности нуля.
\end{Definition}
\begin{Theorem}
  Для непрерывности функции необходима и достаточна бесконечная малость её приращения.
\end{Theorem}
\centerline \it{$f$ непрерывна в точке $x_0 \Leftrightarrow$ приращение бесконечно мало, $\Delta f (x_0)(h) \xrightarrow[h \rightarrow 0]{} 0.$}
\begin{enumerate}
  \item Бывает удобным вместо $h$ использовать символ $\Delta x$:

  \noindent $\Delta f(x_0)(\Delta x) = f(x_0 + \Delta x) - f(x_0).$
  \item Можно рассматривать приращение в форме $\Delta f(x_0)(x) = f(x) - f(x_0).$
\end{enumerate}
\subsection{Точки разрыва}
\begin{Definition}
  Пусть $f$ -- функция, определённая на множестве $E$, $x_0$ -- предельная точка $E$.

  $x_0$ называется точкой разрыва, если $f$ не является непрерывной в этой точке.

  $x_0$ оказывается точкой разрыва, если $f$ не определена в этой точке или $f(x_0)$ не служит пределом для функции $f$.
\end{Definition}
\begin{Definition}[Классификация точек разрыва]
  Пусть $x_0$ -- точка разрыва функции $f$.
\begin{enumerate}
  \item $x_0$ называется точкой разрыва $I$ рода, если функция имеет конечные односторонние пределы $f(x_0 - 0), f(x_0 + 0)$.

  Если при это $f(x_0 - 0) = f(x_0 + 0), x_0$ -- точка устранимого разрыва.

  Если же $f(x_0 - 0) \neq f(x_0 + 0), x_0$ -- точка скачка, число $\Delta = f(x_0 + 0) - f(x_0 - 0)$ называется величиной скачка.
  \item Другие случаи относятся к разрывам $II$ рода.

  Если хотя бы один из односторонних пределов бесконечен, $x_0$ называется точкой бесконечного разрыва.
\end{enumerate}
\end{Definition}
\par\medskip \textbf{Примеры}\par
В точке устранимого разрыва функция имеет конечный предел $A = \lim\limits_{x \rightarrow x_0} f(x)$. Разрыв обусловлен тем, что функция не определена или плохо определена в точке $x_0$. Полагая
$$f(x) = $$
мы получим непрерывную функцию. Говорят, что
