\documentclass{article}
\pagestyle{myheadings}

%%%%%%%%%%%%%%%%%%%
% Packages/Macros %
%%%%%%%%%%%%%%%%%%%
\usepackage{mathrsfs}


\usepackage{fancyhdr}
\pagestyle{fancy}
\lhead{}
\chead{}
\rhead{}
\lfoot{}
\cfoot{}
\rfoot{\normalsize\thepage}
\renewcommand{\headrulewidth}{0pt}
\renewcommand{\footrulewidth}{0pt}
\newcommand{\RomanNumeralCaps}[1]
    {\MakeUppercase{\romannumeral #1}}

\usepackage{amssymb,latexsym}  % Standard packages
\usepackage[utf8]{inputenc}
\usepackage[russian]{babel}
\usepackage{MnSymbol}
\usepackage{mathrsfs}
\usepackage{amsmath,amsthm}
\usepackage{indentfirst}
\usepackage{graphicx}%,vmargin}
\usepackage{epigraph} %%% to make inspirational quotes.
\usepackage[all]{xy} %для XyPic'a
\usepackage{hyperref}
\usepackage{color}
\usepackage{tikz}
\usepackage{amscd} %для коммутативных диграмм
\usepackage{amsmath}


%\renewcommand{\baselinestretch}{1.5}
%\sloppy
%\usepackage{listings}
%\lstset{numbers=left}
%\setmarginsrb{2cm}{1.5cm}{1cm}{1.5cm}{0pt}{0mm}{0pt}{13mm}


\newtheorem{Lemma}{Лемма}[section]
\newtheorem{Proposition}{Предложение}[section]
\newtheorem{Theorem}{Теорема}[section]
\newtheorem{Corollary}{Следствие}[section]
\newtheorem{Remark}{Замечание}[section]
\newtheorem{Definition}{Определение}[section]
\newtheorem{Designations}{Обозначение}[section]




%%%%%%%%%%%%%%%%%%%%%%%%
%Сношение с оглавлением%
%%%%%%%%%%%%%%%%%%%%%%%%
\usepackage{tocloft}
\renewcommand{\cfttoctitlefont}{\hspace{0.38\textwidth} \huge\bfseries\MakeUppercase}
\renewcommand{\cftbeforetoctitleskip}{-1em}
\renewcommand{\cftaftertoctitle}{\mbox{}\hfill \\ \mbox{}\hfill{\footnotesize Стр.}\vspace{-0.5em}}
%\renewcommand{\cftchapfont}{\normalsize\bfseries \MakeUppercase{\chaptername} }
\renewcommand{\cftsecfont}{\hspace{1pt}}
\renewcommand{\cftsubsecfont}{\hspace{1pt}}
%\renewcommand{\cftbeforechapskip}{1em}
\renewcommand{\cftparskip}{3mm} %определяет величину отступа в оглавлении
\renewcommand{\cftdotsep}{1} %частота точек

\setcounter{tocdepth}{3}

%%%%%%Переопределение chapter%%%%%%
\newcommand{\empline}{\mbox{}\newline}
\newcommand{\likechapterheading}[1]{
\begin{center}
\textbf{\MakeUppercase{#1}}
\end{center}
\empline}

%%%%%%%Запиливание переопределённого chapter в оглавление%%%%%%
\makeatletter
\renewcommand{\@dotsep}{2}
\newcommand{\l@likechapter}[2]{{\bfseries\@dottedtocline{0}{0pt}{0pt}{#1}{#2}}}
\makeatother
\newcommand{\likechapter}[1]{
\likechapterheading{#1}
\addcontentsline{toc}{likechapter}{\MakeUppercase{#1}}}



%\renewcommand{\thelikesection}{(\roman{likesection})}
%%%%%%%%%%%
% Margins %
%%%%%%%%%%%
\addtolength{\textwidth}{0.7in}
\textheight=630pt
\addtolength{\evensidemargin}{-0.4in}
\addtolength{\oddsidemargin}{-0.4in}
\addtolength{\topmargin}{-0.4in}


%%%%%%%%%%%%{\arabic{l@likesection}}
% Document %
%%%%%%%%%%%%

%%%%%%%%%%%%%%%%%%%%%%%%%%%%%
%%%%%%главы -- section*%%%%%%
%%%%section -- subsection%%%%
%subsection -- subsubsection%
%%%%%%%%%%%%%%%%%%%%%%%%%%%%%
\def \newstr {\medskip \par \noindent}

\begin{document}

\begin{center}
\section{\LARGE{\bf Предел функции}}
\end{center}
\epigraph{\textit{Ну а это вы изучите сами дома.}}
{-- Моисеев А.А.}
\subsection{Понятие предела функции}
\subparagraph{Предельная точка}
\begin{Definition}
Пусть $E\subseteq\mathbb{R}$ и $a\in\mathbb{R}.$ $a$ называется предельной точкой для $E,$ если $\forall\delta> 0\;\exists x\in E: \; 0<\mid x-a\mid<\delta.$
\end{Definition}

Рассмотрим множество на числовой прямой $(a, b)\cup\{c\}.$ Точки $a$ и $b$ -- предельные точки множества. Точка $c$ называется изолированной точкой множества.



$a$ предельная, если $\forall O$ окрестности $a \; \exists x\in O\cap E, x \neq a.$

Помимо обычной окрестности точки вводят также определение проколотой окрестности: $\dot{O} = O\setminus\{a\}.$

Таким образом, $a$ предельная точка для $E,$ если $\forall O$ -- окрестности точки $a$ выполнено, что $E\cap\dot{O}\neq\varnothing.$

\subparagraph{Определение предела функции}

\begin{Definition}
$E\subseteq\mathbb{R}, a$ -- предельная точка $E, \; f$ -- функция на $\mathbb{E}, A\in\mathbb{R}. \; A$ называется пределом функции $f$ в точке $a, \;f(x)\xrightarrow[x\rightarrow a]{} A; \; A=\lim\limits_{x\rightarrow a} f(x); A=\lim\limits_{a}f,$ если:
\begin{enumerate}
\item на языке $\varepsilon$-$\delta:$

$\forall \varepsilon > 0 \;\exists \delta > 0\;\forall x\in E 0<\mid x-a\mid<\delta \Rightarrow \mid f(x)-A\mid<\varepsilon;$

\item на языке окреcтностей$:$

$\forall O$ окрестности точки $A$ $\exists V$ -- окрестность точки $a: \; f(\dot{V}\cap E \subset O;$

\item на языке последовательностей$:$

$\forall\{x_n\} \; x_n\in E \; x_n \neq a, \; x_n\xrightarrow[n \rightarrow \infty]{} a,  \Rightarrow f(x_n)\xrightarrow[n\rightarrow\infty]{} A.$
\end{enumerate}
\end{Definition}

\par\medskip \textbf{Пример}\par

%КАРТИНКА!!!!!!!!!!!!!!!!!!!!!!!!!!!!!!!!!!!!!!!!!!!!!!!!!!!!!!!!!!!!!!!!!!!!!!!!

Предел -- локальное понятие. Если $f=g$ в $\dot{O}$ проколотой окрестности точки $a,$ то они имеют пределы и они равны.

\subparagraph{Равносильность определений}
\begin{Theorem}[Равносильность определений]
Определения (1), (2) и (3) равносильны. Если $A=\lim\limits_{x \to a} f(x)$ в смысле одного из определений, то $A=\lim\limits_{x \to a} f(x)$ и в смысле других определений.
\end{Theorem}
\begin{proof}
Для начала сравним определения (1) и (2).
Пусть $A=\lim\limits_{x \to a} f(x)$ в смысле (1). Возьмём произвольную окрестность $V$ точки $A$. Найдётся такое $\varepsilon>0$, что $V_\varepsilon(A) \subset V$. По определению (1) существует такое $\delta > 0$, что
$$x \in E, x \neq a, \; \mid x - a \mid \; < \varepsilon \Rightarrow \mid f(x) - A \mid \; < \varepsilon.$$
Полагая $U = V_\delta (a)$, можно переписать предыдущее соотношение в виде $f(U \cap E) \in V_\varepsilon(A) \in V$. Таким образом, $A=\lim\limits_{x \to a} f(x)$ в смысле (2).
Наоборот, пусть $A=\lim\limits_{x \to a} f(x)$ в смысле (2). Возьмём произвольное $\varepsilon > 0$ и положим $V = V_\varepsilon(A)$. По определению (2) найдётся окрестность $U$ точки $a$, для которой $f(U \cap E) \in V$. Подберём $\delta > 0$, для которого $V_\delta(a) \in U$. Теперь
$$\forall x \in E, 0 < \; \mid x - a \mid \; < \delta \Rightarrow \mid f(x) - A \mid<\varepsilon,$$
$A=\lim\limits_{x \to a} f(x)$ в смысле (2). Установлена равносильность определений (1), (2).
Докажем равносильность определений (1), (3).
Пусть $A=\lim\limits_{x \to a} f(x)$ в смысле (1). Возьмём произвольную последовательность
$${x_n}, x_n \neq a, x_n \xrightarrow[n\rightarrow\infty]{} a.$$
и убедимся в том, что $f(x_n) \xrightarrow[n\rightarrow\infty]{} A.$
Возьмём произвольное $\varepsilon > 0$. По определению (1) найдётся такое $\delta > 0$, что
$$\forall x \in E, 0 < \; \mid x - a \mid \; < \delta \Rightarrow \mid f(x) - A \mid < \varepsilon.$$
По определению предела последовательности найдётся номер $N$, для которого
$$ n > N \Rightarrow \mid x_n - a \mid \; < \delta.$$
Видим, что при $n > N$ выполняется соотношение $\mid f(x_n) - A \mid \;  < \varepsilon.$ Итак, $f(x_n) \xrightarrow[n\rightarrow\infty]{} A.$ $A=\lim\limits_{x \to a} f(x)$ в смысле (3).
Пусть $A=\lim\limits_{x \to a} f(x)$ в смысле (3). Допустим, $A$ не является пределом в смысле (1). Тогда
$$\exists\varepsilon > 0 \; \forall \delta > 0 \; \exists x \in E, 0 < \;  \mid x - a \mid \; < \delta, \; \mid f(x) - A \mid \; \geq \varepsilon.$$
В качестве $\delta > 0 $ последовательно возьмём числа $\frac1{n}$, $n = 1, 2, \ldots$ и найдём точки $x_n \in E, 0 < \; \mid x_n - a \mid \; < \frac1{n}, \; \mid f(x_n) - A \mid \; \geq \varepsilon$. Тем самым мы получаем последовательность ${x_n}$. Неравенство $\mid x_n - a \mid < \frac1{n} $ означает, что $x_n \xrightarrow[n\rightarrow\infty]{} a$, а неравенство $\mid f(x_n) - A \mid \geq \varepsilon$ говорит, что $A$ не является пределом для ${f(x_n)}$, вопреки предположению. Полученное противоречие заставляет нас признать, что $A=\lim\limits_{x \to a} f(x)$ в смысле (1).
\end{proof}
\subsection{Различные предельные конструкции}
\subparagraph{Односторонние пределы}
Пусть $f$ определена на некотором интервале $(a - \delta_0, a)$.
Число $A$ -- левосторонний предел функции $f$ в точке $a$ $(A = f(a - 0)$, $A=\lim\limits_{x \to a - 0} f(x), f(x) \xrightarrow[n\rightarrow a-0]{} A),$ если
$$ \forall \varepsilon > 0 \; \exists \delta > 0 \; \forall x: a - \delta < x < a \Rightarrow \; \mid f(x) - A \mid \; < \varepsilon.$$
Пусть $f$ определена на некотором интервале $(a, a + \delta_0)$.
Число $A$ называется правосторонним пределом функции $f$ в точке $a (a = f(a+0), A=\lim\limits_{x \to a+0} f(x)), f(x) \xrightarrow[n\rightarrow a+0]{} A),$ если
$$ \forall \varepsilon > 0 \; \exists \delta > 0 \; \forall x: a < x < a + \delta \Rightarrow \; \mid f(x) - A \mid \; < \varepsilon.$$
Определения односторонних пределов можно дать и в терминах окрестностей и последовательностей.
\begin{Proposition}
Функция $f$ определена в проколотой окрестности точки $a$.
Тогда функция имеет предел в точке $a$ в том и только том случае, если существуют и равны между собой односторонние пределы $f(a-0) = f(a+0).$ В случае существования предела
$$f(a-0) = f(a+0) = \lim\limits_{x \to a} f(x)$$
\end{Proposition}

\subparagraph{Бесконечные пределы}
В определении предела число $A$ можно заменить на $+\infty$, $-\infty$, $\infty$. Например,
$$f(x) \xrightarrow[x\rightarrow a]{} -\infty \Leftrightarrow \forall E > 0 \; \exists \delta > 0, \; 0 < \; \mid x - a \mid \; < \delta \Rightarrow f(x) < -E.$$

\subparagraph{Пределы на бесконечности}
Бесконечности могут выполнять и роль точки, в которой вычисляется предел. Например,
$$f(x) \xrightarrow[x\rightarrow a]{} \infty \Leftrightarrow \forall E > 0 \; \exists \delta > 0: x > \delta \Rightarrow \; \mid f(x) \mid \; > E.$$

\subsection{Простейшие свойства предела функции}
\begin{enumerate}
\item Если функция постоянна в некотрой проколотой окрестности точки $a$, $f(x) = A$ при $x \in V$, то $f(x) \xrightarrow[x\rightarrow a]{} A.$
\item Предел единственен.
\item Если функция имеет конечный предел, то она она ограничена в некотрой проколотой окрестности.
\end{enumerate}

\subsection{Предел и арифметические операции}
\begin{Theorem}
Пусть функции $f, g$ определены в проколотой окрестности $E$ точки $a$.
$$f(x) \xrightarrow[x\rightarrow a]{} A, g(x) \xrightarrow[x\rightarrow a]{} B.$$
Определим функции
$$F: F(x) = f(x) + g(x), x \in E;$$
$$G: G(x) = f(x) \cdot g(x), x \in E;$$
$$H: H(x) = \frac{f(x)}{g(x)}, x \in E.$$
(В последнем случае предполагаем, что $g(x) \neq 0$ при $x \in E$).
Тогда функции $F, G, H$ тоже имеют пределы,
$$F(x) \xrightarrow[x\rightarrow a]{} A + B, G(x) \xrightarrow[x\rightarrow a]{} A \cdot B, H(x) \xrightarrow[x\rightarrow a]{} \frac{A}{B}$$
(последнее при условии $B \neq 0$).
\end{Theorem}

\begin{proof}
Докажем, например, последнее утверждение.
Возьмём произвольную последовательность ${x_n}, x_n \in E, x_n \neq a, x_n \xrightarrow[n\rightarrow \infty]{} B.$
По теореме о пределе отношения последовательностей $H(x_n) \xrightarrow[n \rightarrow \infty]{} \frac{A}{B}.$ Опять по определению предела функции $H(x) \xrightarrow[x \rightarrow a]{} \frac{A}{B}.$
\end{proof}

\subsection{Предел и неравенства}
\begin{Theorem}
Пусть функции $f, g$ определены в проколотой окрестности $E$ точки $a$.

\subparagraph{Стабилизация неравенств}
$$f(x) \xrightarrow[x \rightarrow a]{} A, g(x) \xrightarrow[x \rightarrow a]{} B, A < B.$$
Тогда
$$\exists U \; \forall x \in U, f(x) < g(x).$$

\subparagraph{Предельный переход в неравенстве}
$$\forall x \in E, f(x) < g(x),$$
$$f(x) \xrightarrow[x \rightarrow a]{} A, g(x) \xrightarrow[x \rightarrow a]{} B,$$
Тогда
$$A \leq B.$$
\begin{Remark}
Можно ослабить условие и потребовать выполнения неравенства $f(x) \leq g(x)$ в некоторой проколотой окрестности точки $a$.
\end{Remark}

\subparagraph{Теорема о полицейских}
$$\forall x \in E, f(x) \leq h(x) \leq g(x),$$
$$f(x) \xrightarrow[x \rightarrow a]{} A, g(x) \xrightarrow[x \rightarrow a]{} A.$$
Тогда
$$h(x) \xrightarrow[x \rightarrow a]{} A.$$

\end{Theorem}
\begin{proof}
Рассмотрим окрестности $V_A = \left( -\infty;\frac{A+B}{2} \right)$ и $V_B = \left(\frac{A+B}{2};+\infty \right)$ точек $A$ и $B$ соответственно. На основании определения предела мы можем найти такую окрестность $U$ точки $a$, что при $x \in U$ справедливы включения $f(x) \in V_A$ и $g(x) \in V_B$. $U$ -- искомая окрестность. Поскольку из $f(x) \in V_A$ следует, что $f(x) < \frac{A+B}{2}$, а $g(x) \in V_B$ влечёт $g(x) > \frac{A+B}{2}$, то $f(x) < g(x)$.

Утверждение о предельном переходе в неравенстве доказывается, как и в случае последовательностей, методом от противного. Допустив, что имеет место неравенство $A > B$, мы придём к противоречащему условию выводу о том, что в пределах некоторой проколотой окрестности точки $a$ выполняется неравенство $f(x) > g(x)$.

Возьмём произвольную последовательность ${x_n}, x_n \neq a, x_n \xrightarrow[n \rightarrow \infty]{} a.$

По определению предела на языке последовательностей получаем соотношения $f(x_n) \xrightarrow[n \rightarrow \infty]{} A, g(x_n) \xrightarrow[n \rightarrow \infty]{} A$, а по условию $f(x_n) \leq h(x_n) \leq g(x_n), n = 1, 2, \ldots$ По теореме о полицейских для последовательностей $h(x_n) \xrightarrow[n \rightarrow \infty]{} A$. Вновь пользуясь определением на языке последовательностей, делаем вывод, что $h(x_n) \xrightarrow[n \rightarrow \infty]{} A$.
\end{proof}

\subsection{Бесконечно малые и бесконечно большие функции}
\subparagraph{Бесконечно малые функции}
\begin{Definition}
Функция $\alpha$ -- бесконечно малая при $x \rightarrow a$, если $\alpha(x) \xrightarrow[x \rightarrow a]{} 0$.
\end{Definition}
\begin{Theorem}Сумма бесконечно малых является бесконечно малой.\end{Theorem}
\begin{Theorem}Произведение бесконечно малой на ограниченную функцию является бесконечно малой. \end{Theorem}
\begin{Theorem}[Определение предела в терминах бесконечно малых]
$$f(x) \xrightarrow[x \rightarrow a]{} A \Leftrightarrow \alpha = f - A -- б. м.$$
\end{Theorem}
\subparagraph{Бесконечно большие функции}
\begin{Definition}Функция $f$ называется бесконечно большой ($f(x) \xrightarrow[x \rightarrow a]{} \infty)$, если
$$\forall E > 0 \; \exists \delta > 0, \; 0 < \; \mid x - a \mid \; < \delta \Leftrightarrow \; \mid f(x) \mid \; > E.$$
\end{Definition}

\begin{Theorem} $\alpha$ -- б. м. $\Leftrightarrow f = \frac{1}{\alpha}$ -- б. б. \end{Theorem}

\subsection{Предел монотонной функции}
\begin{Definition}
Пусть $f$ -- функция, определённая на промежутке $\Delta$.
\end{Definition}

\begin{itemize}
\item Функция $f$ называется возрастающей (строго возрастающей), если
$$\forall x, y \in \Delta \; x < y \Rightarrow f(x) \leq f(y) \; \left(f(x) < f(y)\right).$$
\item Функция $f$ называется убывающей (строго убывающей), если
$$\forall x, y \in \Delta \; x < y \Rightarrow f(x) \geq f(y) \; \left(f(x) > f(y)\right).$$
\item Функция называется (строго) монотонной, если она (строго) убывает или (строго) возрастает.
\end{itemize}
\begin{Theorem}
Пусть $f$ -- монотонная функция на интервале $(a, b)$.
Тогда существуют односторонние пределы $f(a+0), f(b-0)$, может быть, бесконечные.
\end{Theorem}
\begin{proof}
Для определённости рассмотрим возрастающую функцию.

Пусть $f$ ограничена сверху. Положим $M = \sup f(a, b)$ и покажем, что $f(x) \xrightarrow[x \rightarrow b-0]{} M$. Возьмём произвольное $\varepsilon > 0$. Найдётся такой $x_0 \in (a, b)$, что $f(x_0) > M - \varepsilon$.
Положим $\delta = b - x_0$. Теперь, $\forall x \in (b - \delta, b)$ имеет место неравенство $x > b - \delta = x_0$, поэтому $f(x) \geq f(x_0) > b - \varepsilon, M - \varepsilon < f(x) \leq M$. Видим, что $f(x) \xrightarrow[x \rightarrow b - 0]{} M$.

В случае неограниченности $f(x) \xrightarrow[x \rightarrow b - 0]{} +\infty$.
Действительно, $\forall E > 0 \; \exists x_0 \; f(x_0) > E$. Далее, $\forall x \in (x_0, b) \; f(x) \geq f(x_0) > E$.
\end{proof}

\subsection{Критерий Коши существования предела}
\begin{Theorem}
Для существования конечного предела $\lim\limits_{x \to \infty} f(x)$ необходимо и достаточно условие Коши
$$\forall \varepsilon > 0 \; \exists \delta > 0, \; 0 < \; \mid x' - x_0 \mid, \; \mid x^n - x_0 \mid \; < \delta \Rightarrow \; \mid f(x^\prime) - f(x^{\prime\prime}) \mid \; < \varepsilon.$$
\end{Theorem}

\begin{proof}
Необходимость. Пусть $f(x) \xrightarrow[x \rightarrow x_0]{} A$.
Возьмём произвольное $\varepsilon > 0$. Найдётся такое $\delta > 0$, что
$$0 < \; \mid x - x_0 \mid \; \delta \Rightarrow \; \mid f(x) - A \mid \; \frac{\varepsilon}{2}.$$
Для любых $x^\prime, x^{\prime\prime}$, удовлетворяющих условиям $0 < \; \mid x^\prime - x_0 \mid, \; \mid x^{\prime\prime} - x_0 \mid \; < \delta$ получается неравенство
$$\mid f(x^\prime) - f(x^{\prime\prime}) \mid \; = \; \mid \left(f(x^\prime) - A\right) - \left(f(x^{\prime\prime}) - A\right) \mid \; \leq \; \mid f(x^\prime) - A \mid + \mid f(x^{\prime\prime}) - A \mid \; < \frac{\varepsilon}{2} + \frac{\varepsilon}{2} = \varepsilon.$$

Достаточность. Возьмём произвольное $\varepsilon > 0$ и подберём $\delta > 0$, о котором идёт речь в условии Коши.

Для произвольной последовательности $\{x_n\}^{\infty}_{n=1}$, для которой $x_n \rightarrow x_0$ и $x_n \neq x_0$, найдётся такой номер $N$, что
$$n > N \Rightarrow 0 < \; \mid x_n - x_0 \mid < \delta.$$
Получается, что
$$\forall n, m > N \; \mid f(x_n) - f(x_m) \mid \; < \varepsilon.$$
Последовательность $\{f(x_n)\}_{n=1}^{\infty}$ фундаментальная, следовательно, она сходится. Осталось убедиться в том, что все такие последовательности имеют один и тот же предел. Пусть последовательности $\{{x^\prime}_n\}, \{{x^{\prime\prime}}_n\}$ имеют предел $x_0$ и состоят из членов, отличных от $x_0$. Построим последовательность ${x_n}$, полагая $x_2n-1 = {x^\prime}_n, x_2n = x^{\prime\prime}_n$. Тогда $x_n \xrightarrow[n \rightarrow \infty]{} x_0$, $x_n \neq x_0$ поэтому существует $A=\lim\limits_{n\rightarrow \infty} f(x_n)$.
По теореме о пределе подпоследовательности: $f(x^\prime), f({x^\prime}_n) \xrightarrow[n \rightarrow \infty]{} A$.
\end{proof}

\subsection{Сравнение функций (бесконечно малых)}
\begin{Definition}
Пусть функции $\alpha, \beta$ определены в некоторой проколотой окрестности точки $x_0, \beta$ не обращается в $0$ ни в одной точке.
\begin{enumerate}
\item Говорят, что $\alpha, \beta$ одного порядка при $x \rightarrow x_0$, если
$$\frac{\alpha(x)}{\beta(x)} \xrightarrow[x \rightarrow x_0]{} A \neq 0, \infty$$
Для бесконечно малых функций $\alpha$, $\beta$ в этом случае говорят, что функция $\alpha\, \beta$ одного порядка малости.
\item $\alpha, \beta$ эквивалентны при $x \rightarrow x_0, \alpha(x) \underset{x \rightarrow x_0}{\sim} \beta(x)$, если
$$\frac{\alpha(x)}{\beta(x)} \xrightarrow[x \rightarrow x_0]{} 1.$$
\begin{Remark} В условиях пункта (1) $\alpha(x) \underset{x \rightarrow x_0}{\sim} A\beta(x).$ \end{Remark}
\item Функция $\alpha$ называется бесконечно малой по сравнению с $\beta, \alpha(x)  \underset{x \rightarrow x_0}{=} o\left(\beta(x) \right)$ ($\alpha$ есть $o$-малое от $\beta$), если
$$\frac{\alpha(x)}{\beta(x)} \xrightarrow[x \rightarrow x_0] 0.$$
Для бесконечно малых функций $\alpha, \beta$ в этом случае говорят, что функция $\alpha$ более высокий порядок малости, чем $\beta$.
В смысле принятого определения запись $\alpha = o$ (1) означает бесконечную малость функции $\alpha$.
\item $\alpha \underset{x \rightarrow x_0}{=} O(\beta)$, если $\frac{\alpha}{\beta}$ ограничено в некоторой проколотой окрестности точки $x_0$.
\end{enumerate}
\par\medskip \textbf{Пример}\par $x + 2x^2 \underset{x \rightarrow 0}{\sim} x, x^2 \underset{x \rightarrow 0}{=} o(x)$.
\end{Definition}

\begin{Definition}
  Если $\alpha(x) \underset{x \rightarrow x_0}{\sim} A \left(\beta(x) \right)^k$, то говорят, что $\alpha$ имеет $k$-й порядок относительно $\beta$.
  Если $\alpha(x) \underset{x \rightarrow x_0}{\sim} A \left(x - x_0 \right)^k$, то $\alpha$ имеет $k$-й порядок малости, функция $A(x-x_0)^k$ называется главной частью б. м. $\alpha$.
  (Для случая $x \rightarrow \infty$ роль основной б. м. выполняет функция $\frac1{x}$, функция $\alpha(x) \underset{x \rightarrow x_0}{\sim} \frac{A}{x^k}$ имеет $k$-й порядок малости, имеет $\frac{A}{x^k}$ своей главной частью).
\end{Definition}

\begin{Theorem}
При вычислении пределов сомножители можно заменять на эквивалентные.

Пусть $\alpha(x) \underset{x \rightarrow x_0}{\sim} \alpha_1(x), \beta(x) \underset{x \rightarrow x_0}{\sim} \beta_1(x)$.

Тогда
\begin{enumerate}
  \item Если $\frac{\alpha_1(x)}{\beta_1(x)} \xrightarrow[x \rightarrow x_0]{} A$, то $\alpha(x)\beta(x) \xrightarrow[x \rightarrow x_0]{} A$.
  \item Если $\alpha_1(x) \beta_1(x) \xrightarrow[x \rightarrow x_0]{} A$, то $\frac{\alpha(x)}{\beta(x)} \xrightarrow[x \rightarrow x_0]{} A$.
  \item $\frac{\alpha(x)}{\beta(x)} \underset{x \rightarrow x_0}{\sim} \frac{\alpha_1(x)}{\beta_1(x)}$.
  \item $\alpha(x) \beta(x) \underset{x \rightarrow x_0}{\sim} \alpha_1(x) \beta_1(x)$.
\end{enumerate}
\end{Theorem}

\begin{proof}
\begin{enumerate}
  \item $\frac{\alpha(x)}{\beta(x)} = \frac{\alpha_1(x)}{\beta_1(x)} \frac{\alpha(x)}{\alpha_1(x)} \frac{\beta_1(x)}{\beta(x)} \xrightarrow[x \rightarrow x_0]{} A \cdot 1 \cdot 1 = A$.
  \item
  \item
  \item $\frac{\alpha(x) \beta{x}}{\alpha_1(x) \beta_1{x}} = \frac{\alpha(x)}{\alpha_1(x)} \frac{\beta(x)}{\beta_1(x)} \xrightarrow[x \rightarrow x_0]{} = 1 \cdot 1 = 1.$
\end{enumerate}
\end{proof}

\begin{Theorem}[Условие эквивалентности]
$$\alpha \sim \beta \Leftrightarrow \alpha - \beta = o(\beta)$$
$$\alpha \sim \beta \Leftrightarrow \alpha = \beta + o(\beta)$$
\end{Theorem}

\begin{proof}
$$\alpha \sim \beta \Leftrightarrow \frac{\alpha}{\beta} \rightarrow 1 \Leftrightarrow \frac{\alpha}{\beta} - 1 \rightarrow 0 \Leftrightarrow \frac{\alpha - \beta}{\beta} \rightarrow 0 \Leftrightarrow \alpha - \beta = o(\beta).$$
\end{proof}


\subparagraph{Операции с $o$}
\begin{enumerate}
  \item Если $\beta_1, \beta_2$ одного порядка, то $o(\beta_1) = o(\beta_2)$.
  \item $o(\beta) \pm o(\beta) = o(\beta)$.
  Равенство следует понимать как утверждение:
  $$Если \alpha_1 = o(\beta) и \alpha_2 = o(\beta), то \alpha_1 \pm \alpha_2 = o(\beta).$$
  \item $\gamma \cdot o(\beta) = o(\gamma \beta), o(\gamma) \cdot o(\beta) = o(\gamma \beta).$
  \item Если $\alpha = o(\beta), \beta = o(\gamma)$, то $\alpha = o(\gamma)$.
  Действительно, $\frac{\alpha}{\gamma} = \frac{\alpha}{\beta} \frac{\beta}{\gamma} \rightarrow 0 \cdot 0 = 0$.
\end{enumerate}
\par\medskip \textbf{Пример}\par $x^3 \underset{x \rightarrow 0}{=} o(x^2), x^2 \underset{x \rightarrow 0}{=} o(x)$, поэтому $x^3 \underset{x \rightarrow 0}{=} o(x)$, иными словами $o(x^2) \underset{x \rightarrow 0}{=} o(x)$. Отметим, что в последнем "равенстве" нельзя менять местами левую и правую части.

\end{document}
