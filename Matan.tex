\documentclass{article}
\pagestyle{myheadings}

%%%%%%%%%%%%%%%%%%%
% Packages/Macros %
%%%%%%%%%%%%%%%%%%%
\usepackage{mathrsfs}


\usepackage{fancyhdr}
\pagestyle{fancy}
\lhead{}
\chead{}
\rhead{}
\lfoot{}
\cfoot{} 
\rfoot{\normalsize\thepage}
\renewcommand{\headrulewidth}{0pt}
\renewcommand{\footrulewidth}{0pt}
\newcommand{\RomanNumeralCaps}[1]
    {\MakeUppercase{\romannumeral #1}}

\usepackage{amssymb,latexsym}  % Standard packages
\usepackage[utf8]{inputenc}
\usepackage[russian]{babel}
\usepackage{MnSymbol}
\usepackage{mathrsfs}
\usepackage{amsmath,amsthm}
\usepackage{indentfirst}
\usepackage{graphicx}%,vmargin}
\usepackage{epigraph} %%% to make inspirational quotes.
\usepackage[all]{xy} %для XyPic'a
\usepackage{hyperref}
\usepackage{color} 
\usepackage{amscd} %для коммутативных диграмм

%\renewcommand{\baselinestretch}{1.5}
%\sloppy
%\usepackage{listings}
%\lstset{numbers=left}
%\setmarginsrb{2cm}{1.5cm}{1cm}{1.5cm}{0pt}{0mm}{0pt}{13mm}


\newtheorem{Lemma}{Лемма}[section]
\newtheorem{Proposition}{Предложение}[section]
\newtheorem{Theorem}{Теорема}[section]
\newtheorem{Corollary}{Следствие}[section]
\newtheorem{Remark}{Замечание}[section]
\newtheorem{Definition}{Определение}[section]
\newtheorem{Designations}{Обозначение}[section]




%%%%%%%%%%%%%%%%%%%%%%%% 
%Сношение с оглавлением% 
%%%%%%%%%%%%%%%%%%%%%%%% 
\usepackage{tocloft} 
\renewcommand{\cfttoctitlefont}{\hspace{0.38\textwidth} \huge\bfseries\MakeUppercase} 
\renewcommand{\cftbeforetoctitleskip}{-1em} 
\renewcommand{\cftaftertoctitle}{\mbox{}\hfill \\ \mbox{}\hfill{\footnotesize Стр.}\vspace{-0.5em}} 
%\renewcommand{\cftchapfont}{\normalsize\bfseries \MakeUppercase{\chaptername} } 
\renewcommand{\cftsecfont}{\hspace{1pt}} 
\renewcommand{\cftsubsecfont}{\hspace{1pt}} 
%\renewcommand{\cftbeforechapskip}{1em} 
\renewcommand{\cftparskip}{3mm} %определяет величину отступа в оглавлении
\renewcommand{\cftdotsep}{1} %частота точек

\setcounter{tocdepth}{3} 

%%%%%%Переопределение chapter%%%%%% 
\newcommand{\empline}{\mbox{}\newline} 
\newcommand{\likechapterheading}[1]{ 
\begin{center} 
\textbf{\MakeUppercase{#1}} 
\end{center} 
\empline} 

%%%%%%%Запиливание переопределённого chapter в оглавление%%%%%% 
\makeatletter 
\renewcommand{\@dotsep}{2} 
\newcommand{\l@likechapter}[2]{{\bfseries\@dottedtocline{0}{0pt}{0pt}{#1}{#2}}} 
\makeatother 
\newcommand{\likechapter}[1]{ 
\likechapterheading{#1} 
\addcontentsline{toc}{likechapter}{\MakeUppercase{#1}}} 


   
%\renewcommand{\thelikesection}{(\roman{likesection})}
%%%%%%%%%%%
% Margins %
%%%%%%%%%%%
\addtolength{\textwidth}{0.7in}
\textheight=630pt
\addtolength{\evensidemargin}{-0.4in}
\addtolength{\oddsidemargin}{-0.4in}
\addtolength{\topmargin}{-0.4in}


%%%%%%%%%%%%{\arabic{l@likesection}}
% Document %
%%%%%%%%%%%%

%%%%%%%%%%%%%%%%%%%%%%%%%%%%%
%%%%%%главы -- section*%%%%%%
%%%%section -- subsection%%%%
%subsection -- subsubsection%
%%%%%%%%%%%%%%%%%%%%%%%%%%%%%
\def \newstr {\medskip \par \noindent} 



\begin{document}
\thispagestyle{empty}
\begin{center} 
\rule[0.5ex]{\linewidth}{2pt}\vspace*{-\baselineskip}\vspace*{3.2pt} 
\rule[0.5ex]{\linewidth}{1pt}\\[\baselineskip] 
{\huge \sc Конспект по математическому}\\[4mm] 
{\huge \sc анализу для первого курса}\\[4mm] 
\rule[0.5ex]{\linewidth}{1pt}\vspace*{-\baselineskip}\vspace{3.2pt} 
\rule[0.5ex]{\linewidth}{2pt}\\ 
\vspace{6.5mm} 
 
\vspace{4mm} 
{\large СПбПУ, Институт ИПММ\\ 
\smallskip
\smallskip
\textsc{Бакалавриат "Математика и CS"}}\\ 

\vspace{6.5mm} 
{\large\textsc{Лекции Моисеева Алексея Александровича}}\\ 
\vspace{5mm} 

\vspace{20mm} 

\end{center} 
\author{Я}
\begin{center}
\vfill {\large\textsc{Санкт-Петербург}}\\ 
 \today
\end{center}
%%%%%%%%%%%%%%%%%%%%%%%%%%%%%%%%%%%%%%%%%%%%%%%%%%%%%%%%%%%%%%%%%%%%%%%%%%%%%%%%%%%%%%%%%%%%%%
%\ \\[4cm]

%\begin{center}
%\LARGE Конспект по математическому анализу за первый семестр бакалавриата "Математика и CS" %\quad СПбГПУ (лекции Моисеева Алексея Александровича)
%\end{center}

%\ \\[11cm]

%by Плаксин Д.А.
%\rm
%%%%%%%%%%%%%%%%%%%%%%%%%%%%%%%%%%%%%%%%%%%%%%%%%%%%%%%%%%%%%%%%%%%%%%%%%%%%%%%%%%%%%%%%%%%%%%

\newpage

\tableofcontents{}

\newpage
\pagestyle{plain}
\begin{center}
\large \bf Упоминаемая литература
\end{center}

\smallskip

\begin{enumerate}
\item А.П.Аксёнов "Математический анализ, часть 1"
\item В.М.Зорич "Математический анализ, часть 1"
\item Б.П.Демидович "Сборник задач и упражнений по математическому анализу"
\end{enumerate}

\newpage 
\begin{center}

\smallskip
\smallskip
\smallskip
\smallskip

\section{\LARGE{\bf Введение}}


\end{center}

\epigraph{\textit{Будет солнечно или мы пойдём на лекцию?}}
{-- Моисеев А.А.}

\subsection{Логическая символика}

\begin{itemize}
\item Операция $"\neg"$ называется отрицанием и, если $A$ -- суждение, то $\neg A$ есть отрицание этого суждения.
\item Операция $"\vee"$ называется дизъюнкцией, и под $A \vee B$(\textit{читается $"A$ или $B"$}) подразумевается, что выполнено хоть одно из суждений $A$ или $B.$

\item Операция $"\&"$ $("\wedge")$ называется конъюнкцией и $A\;\&\;B("A$ и $B")$ значит что имеют место быть оба высказывания $A$ и $B$ одновременно.

\item Символ $"\Rightarrow"$ называется импликацией и запись $A \Rightarrow B$ изначает, что из высказывания $A$ следует высказывание $B,$ при этом $B$ есть необходимое условия для $A$, а $A$ является достаточным условием для $B.$

\item Символ $"\Leftrightarrow"$ называется равносильностью и $A \Leftrightarrow B$ значит, что для высказывания $A$ необходимо и достаточно условие $B.$

\item Квантор всеобщности $\forall$ читается как $"$для любого/для всякого$"$ и означает, соответственно, то же самое.

\item Квантор существования $\exists$ читается как $"$существует/найдётся$"$.
\end{itemize}

{\bf Важные соотношения:}

\begin{enumerate}
\item $\neg(A\vee B) \Leftrightarrow \neg A \& \neg B$
\item $\neg(A\;\&\; B)\Leftrightarrow \neg A\vee\neg B$
\item $A \Rightarrow B  \Leftrightarrow \neg A\Rightarrow\neg B$
\item $A\Rightarrow B  \Leftrightarrow\neg A\vee B$
\item $\neg \left(\forall x\right) P(x) \Leftrightarrow \exists x \;\;\neg P(x)$
\item $\neg \left(\exists x\right) P(x) \Leftrightarrow \forall x \;\;\neg P(x)$
\end{enumerate}

\subsection{Множества и операции над ними}

\begin{Definition}
Множество -- первичное понятие, обозначается заглавной буквой. Множества состоят из элементов(определённых и хорошо различимых [определение по Кантору]).
\end{Definition}

Если элемент $x$ принадлежит множеству $A,$ тогда это записывается как $x\in A$ и читается $"x$ принадлежит/из $A".$

Аналогично $y\notin A.$

Множество полностью определяется своими элементами. Два множества являются равными тогда и только тогда, когда все их элементы совпадают.\\ $A=B \Leftrightarrow \forall x\;\; x\in A \Rightarrow x\in B.$

\begin{itemize}
\item {\bf Отношение включения}\\
$A\subseteq B$ \textit{ ("$A$ включено в $B$")} $\Leftrightarrow \forall x\;\; x\in A \Rightarrow x\in B.$

\item {\bf Объединение множеств}\\
Пусть $A, B$ -- множества. Тогда множество $C=\{x\mid x\in A \vee x\in B\}$ называется объединением множеств $A$ и $B$ и записывается $C=A\cup B.$

\item {\bf Пересечение множеств}\\
Пусть $A, B$ -- множества. Тогда множество $C=\{x\mid x\in A\;\&\; x\in B\}$ называется пересечением множеств $A$ и $B$ и обозначается $C=A\cap B.$\\
\\Заметим, что если $A$ и $B$ не имеют общих элементов, то тогда их пересечение равно пустому множеству: $C=A\cap B=\varnothing.$
\end{itemize}

Операции объединения и пересечения коммутативны и ассоциативны и связаны между собой дистрибутивными законами.
$$\left(A\cup B\right)\cap C = \left(A\cap B\right)\cup\left(B\cap C\right)$$
$$\left(A\cap B\right)\cup C = \left(A\cup B\right)\cap\left(B\cup C\right)$$
\begin{proof}
Докажем первое равенство. \\
Покажем, что если $x$ принадлежит $\left(A\cup B\right)\cap C$, то $x$ принадлежит $\left(A\cap B\right)\cup\left(B\cap C\right).$ Пусть $x\in \left(A\cup B\right)\cap C,$ тогда $x\in \left(A \cup B\right) \;\&\;x\in C$, отсюда $\left( x\in A \vee x\in B\right) \;\&\; x\in C = 
\\=\left(x\in A \;\&\; x\in C\right) \vee \left(x\in B \;\&\; x\in C\right) = \left(A\cap B\right)\cup\left(B\cap C\right),$ по дистрибутивному свойству операций $\vee,\&.$\\
Второе неравенство доказывается аналогично.
\end{proof}
\subparagraph{Кое-какие свойства}
\begin{enumerate}
\item $A\cup B = A \Leftrightarrow B\subseteq A$
\begin{proof}
$\Rightarrow:$
\\Возьмём элемент из $B.$ Докажем, что он принадлежит $A.$ Пусть $x\notin A.$ По условию $A\cup B = \{x\mid x\in A \vee x\in B\} = \{x\mid x\in A\} = A,$ то есть $x\in A,$ противоречие.\\
$\Leftarrow:$\\
Докажем равенство двумя включениями:

$\subseteq:$

Возьмём $x\in A\cup B.$ По определению объеднинения $x$ точно лежит в одном из двух множеств. Если он попал в $A$, то всё доказано. Если он попал в $B$, то, так как $B\subseteq A,$ то $x\in A.$ Таким образом $A\cup B\subseteq A.$

$\supseteq:$

Возьмём $x\in A,$ докажем, что $x$ точно будет содержаться в объединении $A\cup B.$ Очевидно, что это выполнено.
\end{proof}
\item $A\cap B = A \Leftrightarrow A\subseteq B$
\begin{proof}
$\Rightarrow:$

Возьмём $x\in A$ и докажем, что он принадлежит $B.$ Так как их того, что $x\in A \; \& \; x\in B$ следует, что $x\in A,$ то для любого $x\in A$ следует то, что $x\in B,$ тем самым, $A\subset B.$

$\Leftarrow:$

Так же докажем двумя включениями:

$\subseteq:$

Возьмём $x\in A\cap B \Rightarrow x\in A \; \& \; x\in B,$ т.к. $A\subseteq B,$ то $x\in A.$

$\supseteq:$

Возьмём $x\in A,$ так как $A\subseteq B,$ то $x\in A \; \& \; x\in B \Rightarrow x\in A\cap B.$
\end{proof}
\end{enumerate}

\begin{itemize}
\item {\bf Разность множеств}
\\$A\setminus B = \{x\mid x\in A \;\&\; x\notin B\}$
\\Если $A\subseteq M,$ тогда $M\setminus A$ называется дополнением $A$ в $M$ и обозначается $cA.$\\
\\Операция дополнения связана с операциями пересечения и объединения следующими соотношениями, называющимися формулами двойственности де Моргана: 
$$c\left(A\cup B\right)=cA\cap cB$$
$$c\left(A\cap B\right)=cA\cup cB$$

\begin{proof}
Докажем первое равенство. 
\begin{eqnarray}
x\in c\left(A\cup B\right) \Leftrightarrow x\notin A\cup B \Leftrightarrow \neg\left(x\in A \vee x\in B\right) \Leftrightarrow x\notin A \;\&\; x\notin B \Leftrightarrow x\in cA \;\&\; x\in cB \Leftrightarrow \nonumber \\ 
\Leftrightarrow x\in cA\cap cB.\nonumber
\end{eqnarray}
Второе неравенство доказывается аналогично.
\end{proof}

\item {\bf Декартово произведение множеств(прямое произведение)}
\\Пусть $X$ и $Y$ непустые множества, тогда их декартово произведение определяется следующим образом: $X\times Y = \{(x,y)\mid x\in X, y\in Y\},$ если хоть одно из множеств пусто, тогда всё декартово произведение будет считаться пустым.

\begin{Remark}
Таким образом плоскость есть декартово произведение осей.
\end{Remark}
\end{itemize}

\subsection{Отображения(Функции)}

\begin{Definition}
Пусть $X$ и $Y$ -- не пустые множества, тогда правило-закон $f,$ который каждому $x\in X$ однозначно сопостовляет некий $y\in Y,$ называется отображением(функцией) определённой на $X$ со значениями в $Y$ и обозначается: $f:X \rightarrow Y$ или $X \xrightarrow{\text{f}}Y$
\end{Definition}

Элемент $y\in Y,$ соответсвующий $x\in X$ называется значением отображения $f$ в точке $x$ и записывается: $y=f(x).$

Отображение $f$ называется функцией, когда $Y$ есть числовое множество.

Если имеется функция $f:X \rightarrow Y,$ то тогда множество $X$ называется областью определения, а $Y$ областью прибытия функции $f.$

\subparagraph{Примеры:}
\begin{enumerate}
\item Пусть область определения совпадает с областью прибытия и равна множеству вещественных чисел $X=Y=\mathbb{R},$ и функцию $f:X \rightarrow Y.$ Как видно из примера, область прибытия не всегда совпадает с множеством значений.
\item Проектирование декартова произведения множеств.
\\Пусть множество $X$ является декартовым произведением двух множеств $X_1$ и $X_2:$ $X=X_1 \times X_2.$ Рассмотрим две функции: $$pr_1:X_1\times X_2 \rightarrow X_1$$
$$(x_1,x_2) \rightarrow x_1$$
и $$pr_2:X_1\times X_2 \rightarrow X_2$$
$$(x_1,x_2) \rightarrow x_2,$$
тогда $pr_1$ называется проекцией множества $X_1\times X_2$ на $X_1$, а $pr_2$, соответственно, проекцией на $X_2.$
\item Пусть $X$ -- множество, тогда $\mathcal{P}(X)$ является множеством всех подмножеств(универсумом) множества $X.$

Можно рассмотреть следующее отображение: $$f:\mathcal{P}(X) \rightarrow \mathcal{P}(X),$$ которое множеству $A\subseteq X$ сопостоставляет его дополнение в $X:$ $X\setminus A.$
\end{enumerate}

\begin{Definition}
Пусть имеется функция $f:X \rightarrow Y,$ тогда {\bf сужением функции} $f$ на некое подмножество $E\subseteq X$ называется функция $g:E\rightarrow Y$ такая, что $g(x) = f(x), \forall x\in E$ и обозначается как $f|_{E}.$
\end{Definition}

\begin{Definition}
{\bf Образом множества} $A\subseteq X$ при отображении $f:X \rightarrow Y$ называют такое множество $f(A) = \{y\in Y\mid \exists x\in A: f(x)=y\} = \{f(x)\mid x\in A\}.$
\end{Definition}

\begin{Definition}
{\bf Праобразом множества} $B\subseteq Y$ при отображении $f:X\rightarrow Y$ называется множество $f^{-1}(B) = \{x\in X\mid f(x)\in B\}.$
\end{Definition}

\begin{Remark}
$$f^{-1}\left(f(A)\right)\supseteq  A$$
$$f\left(f^{-1}(B)\right) \subseteq B$$
\end{Remark}

\subparagraph{Инъекция и Сюръекция}
\begin{Definition}
Имеется функция $f:X \rightarrow Y.$ Если $f(X)=Y,$ то $f$ -- сюръекция.
\end{Definition}
\begin{Definition}
Если имеется функция $f:X \rightarrow Y$ и $\forall x_1, x_2 \in X \; x_1 \neq x_2 \Rightarrow f(x_1)\neq f(x_2),$ то тогда функция $f$ называется инъекцией.
\end{Definition}
\begin{Definition}
$f$ -- биекция, если $f$ сюръекция и инъекция.
\end{Definition} 

\subparagraph{Композиция отображений}
Пусть имеются две функции: $f:X \rightarrow Y$ и $g:Y \rightarrow Z.$
Определим отображение $h:X \rightarrow Z$ следующим образом: $\forall x\in X \; h(x)=g(f(x)),$ тогда $h=g\circ f$ и называется композицией $f$ и $g.$

%КОММУТАТИВНАЯ ДИАГРАММА!!!!!!!!!!!!!!!!!!!!!!!!!!!!!!!!!!!!!!!!!!!!!!!!!!!!!!!!!
\subparagraph{Обратое отображение}
%!!!!!!!!!!!!!!!!!!!!!!!!!!!!!!!!!!!!!!!!!!!!!!!!!!!!!!!!!!!!!!!!!!!!!!!!!!!!!!!!!!!!!!!!!!!!!
\subparagraph{График отображения}
%КАРТИНКА!!!!!!!!!!!!!!!!!!!!!!!!!!!!!!!!!!!!!!!!!!!!!!!!!!!!!!!!!!!!!!!!!!!!!!!!

%!!!!!!!!!!!!!!!!!!!!!!!!!!!!!!!!!!!!!!!!!!!!!!!!!!!!!!!!!!!!!!!!!!!!!!!!!!!!!!!!!!!!!!!!!!!!!
\subsection{Вещественные  числа}
$\mathbb{R}$ -- множество вещественных чисел.

Аксиоматическое определение множества вещественных чисел:\\

{\bf \RomanNumeralCaps{1}.} в $\mathbb{R}$ введена операция сложения: $(x,y) \xrightarrow{\text{+}} z=x+y.$
\par Свойства действия сложения:
\begin{enumerate}
\item коммутативность \quad  $x+y=y+x,$ $\forall x, y ;$
\item ассоциативность \quad  $(x+y)+z=x+(y+z)=x+y+z,$ $\forall x, y, z ;$
\item существование нейтрального элемента (нуля) \quad $\forall x\;\;$ $x+0=x;$
\item существование противаположного элемента \quad $\forall x\;\;$  $\exists ! -x\;\;$ $x+(-x)=0.$
\end{enumerate}

{\bf \RomanNumeralCaps{2}.} В $\mathbb{R}$ введено умножение. $\cdot:(x,y)\rightarrow z=x\cdot y.$
\begin{enumerate}
\item коммутативность \quad $xy=yx;$
\item ассоциативность \quad $x(yz)=(xy)z=xyz;$
\item существование нейтрального элемента (единицы) \quad $1x=x;$
\item существование обратного элемента \quad $\forall x \; \exists x^{-1}: \; xx^{-1}=1.$
\end{enumerate}

{\bf \RomanNumeralCaps{3}.} Дистрибутивная связь. $(x+y)z=xz+yz.$

{\bf \RomanNumeralCaps{4}.} Отношение порядка. 

Некоторые пары $x,y\in\mathbb{R}$ удовлетворяют условию $x\leq y.$

Свойства отношения $\leq:$
\begin{enumerate}
\item рефлексивность \quad $x\leq x;$
\item антисимметричность \quad $x\leq y \& y\leq x \Rightarrow x=y;$
\item транзитивность \quad $x\leq y \& y\leq z \Rightarrow x\leq z.$
\end{enumerate}
-- общие свойства порядка.\\

{\bf \RomanNumeralCaps{5}.} Связь операций и порядка.

$x\leq y \Rightarrow x+z\leq x+z;$

$x\leq y \Rightarrow xz\leq yz.$

{\bf \RomanNumeralCaps{6}.} Аксиома полноты.

Пусть $X, Y\subset \mathbb{R}$ и $\forall x\in X \; \forall y\in Y \; x\leq y.$ Тогда существует $c\in\mathbb{R}: \; \forall x\in X \; \forall y\in Y \; x\leq c\leq y.$
%КАРТИНКА!!!!!!!!!!!!!!!!!!!!!!!!!!!!!!!!!!!!!!!!!!!!!!!!!!!!!!!!!!!!!!!!!!!!!!!!
\begin{Definition}

\end{Definition}


\subsection{Промежутки в множестве $\mathbb{R}$}
Промежуток есть множество одного из следующих типов:
\begin{enumerate}
\item $[a,b] =\{x\in\mathbb{R}\mid a\leq x\leq b\}$ -- отрезок
\item $(a,b) =\{x\in\mathbb{R}\mid a< x< b\}$ -- интервал
\item $[a,b) =\{x\in\mathbb{R}\mid a\leq x< b\}$ -- полуинтервал
\item $(a,b] =\{x\in\mathbb{R}\mid a< x\leq b\}$ -- полуинтервал
\item $[a, +\infty] =\{x\in\mathbb{R}\mid x\geq a\}$ -- луч
\item $[a,+\infty) =\{x\in\mathbb{R}\mid x>a\}$ -- луч
\item $[-\infty,b] =\{x\in\mathbb{R}\mid x\leq a\}$ -- луч
\item $[-\infty,b) =\{x\in\mathbb{R}\mid x<a\}$ -- луч
\item $[-\infty,+\infty) =\mathbb{R}$ 
\end{enumerate}
\subsection{Геометрическая интерпретация}
$\mathbb{R}$ изображается прямой с отмеченными точками $0$ и $1.$
%!!!!!!!!!!!!!!!!!!!!!!!!!!!!!!!!!!!!!!!!!!!!!!!!!!!!!!!!!!!!!!!!!!!!!!!!!!!!!!!!!!!!!!!!!!!!!
\subsection{Абсолютная величина вещественного числа}

\subsection{Натуральные числа}

\subsection{Целые числа}

\subsection{Рациональные числа}

\begin{Definition}
Множество рациональных чисел обозначается как $\mathbb{Q}$ и состоит из элементов вида $\frac{m}{n},$ где $n\in \mathbb{N}$ и $m\in\mathbb{Z}.$
\end{Definition}

\begin{Definition}
Множество $X$ называется плотным во множестве $Y$ \quad $\Leftrightarrow$ \quad $\Leftrightarrow$ $\forall \alpha, \beta \in \mathbb{Y}:$ $\alpha<\beta \quad \exists x\in X: x\in(\alpha,\beta.)$
\end{Definition}

\begin{Theorem}
$\mathbb{Q}$ плотно в $\mathbb{R}.$
\end{Theorem}
\begin{proof}
Возьмём $\varepsilon = \beta - \alpha > 0.$ По принципу Архимеда $\exists n\in \mathbb{N}: \; \frac{1}{n}<\varepsilon$ и $\exists m\in\mathbb{Z}: \; \frac{m}{n} >\alpha.$ Среди таких чисел выберем наименьшее $m$ такое, что $\frac{m}{n}>\alpha,$ тогда $\frac{m-1}{n}\leq \alpha.$ Обозначая $x=\frac{m}{n}$ получаем, что $X\in(\alpha,\beta).$ Ну, действительно, известно, что $\frac{m-1}{n}<\alpha,$ тогда $\frac{m}{n}\leq\alpha+\frac{1}{n},$ то есть $x\leq \alpha+
\frac{1}{n}<\alpha+\varepsilon=\beta,$ откуда $\alpha < \beta.$
\end{proof}

\begin{Proposition}
Любое вещественное число можно с любой степенью точности приблизить рациональными числами. 
\end{Proposition}
\begin{proof}
Возьмём $x\in(0,1].$ Среди десятичных дробей: $0.1, 0.2,\ldots, 0.9$ выберем наибольшую из тех, что меньше $x.$ Пусть эта дробь равна $0.\alpha_1.$ Теперь возьмём $x_1$ -- меньшее $x,$ но $x_1+\frac{1}{10}>x.$
\smallskip

Среди дробей: $0.\alpha_1 1, 0.\alpha_1 2,\ldots, 0.\alpha_1 9$ выберем $x_2=0.\alpha_1\alpha_2: x_2<x<x_2+\frac{1}{100}.$ 
\smallskip

Таким образом получаем последовательность десятичных дробей $x_n=0.\alpha_1\alpha_2\ldots\alpha_n: x_n<x<x_n+\frac{1}{10^n}.$ Тогда бесконечная дробь $x=0.\alpha_1\alpha_2\ldots\alpha_n\ldots$ представляет вещественное число $x.$
\end{proof}

В $\mathbb{Q}$ выполнены все арифетические действия.
\subsection{Мощность}
\begin{Definition}
Множества $X$ и $Y$ равномощны, если $\exists f:X\rightarrow Y$ -- инъективное отображение. Записывается как $X\sim Y; \mid X\mid=\mid Y\mid$ или как $cardX=cardY.$ (\textit{кардинальное число $X$})
\end{Definition}

Заметим, что если $X\subseteq Y,$ то $\mid X\mid \leq\mid Y\mid.$

\begin{Proposition}
Множество натуральных чисел $\mathbb{N}$ равномощно множеству чётных чисел $\mathbb{N}_{2n}.$
\end{Proposition}
\begin{proof}
Ну, действительно, возьмём функцию $f(n)=2n.$ Легко понять, что $f$ биекция, значит, по определению, множества равномощны.
\end{proof}

\begin{Theorem}[Шредер-Бернштейн]
Если $\mid X\mid\leq\mid Y\mid$ и $\mid Y\mid\leq X\mid,$ то $\mid X\mid = \mid Y\mid.$
\end{Theorem} 

Любые два множества сравнимы по мощности: $\mid X\mid<\mid Y\mid\vee \mid X\mid=\mid Y\mid\vee \mid X\mid>\mid Y\mid.$ 

\begin{Theorem}[Кантора]
Пусть $X$ -- множество, $\mathcal{P}(X)$ -- множество всех подмножеств $X.$ Тогда $\mid\mathcal{P}(X)\mid >\mid X\mid.$
\end{Theorem}
\begin{proof}
Рассмотрим отображение $X\rightarrow\mathcal{P}(X)$ такое, что $x\in X \rightarrow\{x\}\in\mathcal{P}(X)$ -- инъекция, значит $\mathcal{P}(X)\mid\geq\mid X\mid.$ Теперь покажем, что $\mid \mathbb{P}(X)\mid \neq\mid X\mid:$\\

Предположим, что $\mathcal{P}(X)\mid=\mid X\mid,$ тогда $f:X \rightarrow \mathcal{P}(X)$ -- сюръекция, тогда $\forall y \in Y \; Y=f(y)\in\mathcal{P}(X), Y\subseteq X.$ Рассмотрим $Y_0=\{y\in X\mid y\notin f(y)\},$  $Y_0\subseteq X,$ значит $Y_0\in\mathcal{P}(X).$ Так как $f$ сюръекция, то $\exists y_0 \in Y_0: \; f(y_0)=Y_0.$\\

Будет ли верно, что $y_0\in Y_0?$ Ну, проверим: пусть $y_0\in Y_0,$ следовательно $y_0 \neq f(y_0)=Y$ -- противоречие, значит $y_0 \notin Y_0,$ но $Y_0 = f(y_0),$ откуда $y_0\in Y_0$ -- противоречие, значит $f$ не сюръекция.
\end{proof}
\smallskip
\smallskip
\subsubsection{Конечное множество}
Пусть $n\in\mathbb{N},$ тогда $\mathbb{N}_n=\{m\in\mathbb{N}\mid m\leq n\}=\{1, 2,\ldots, n\}$ -- отрезок натурального ряда.
\begin{Definition}
Если $X\sim \mathbb{N}_n,$ то $X$ называется конечным множеством и $n=card X$ -- число элементов $X.$
\end{Definition}

Если $n\neq m,$ то $\mid\mathbb{N}_n\neq\mathbb{N}_m.$

\begin{Proposition}
Множество $X$ конечно $\Leftrightarrow$ $X$ не равномощно своей правильной части$.$ $\mid X\mid\neq\mid Y\mid$, при $Y\subseteq X$ и $Y\neq Y.$
\end{Proposition}

Рассмотрим отображение из $\mathbb{N}$ в $\mathbb{N}\setminus\{1\}$ такое, что $n \rightarrow n+1.$ Так как отображение биективно, то множество $\mathbb{N}\setminus\{1\}$ не конечно.
\begin{Definition}
Бесконечное множество -- не конечное множество.
\end{Definition}

\subsubsection{Счётное множество}
Пусть $X$ бесконечное множество, тогда, если $\forall n \; \exists x_1,\ldots, x_n \in X$ -- попарно различны, то $\exists x_1,\ldots, x_n, x_{n+1} \in X$ -- попарно различны.\\

Тогда $x_1,\ldots$ образуют множество, равновощное множеству натуральных чисел. Таким образом $\mathbb{N}$ самое маленькое бесконечное множество.

\begin{Definition}
Если множество $X$ счётно, то пишут, что его мощность $cardX=\aleph_0$ (\textit{Алеф ноль})$.$
\end{Definition}

Если множество $X$ конечно или счётно, то $X$ называется не более чем счётным множеством.

Пусть $X$ счётно, тогда $\mathcal{X}$ несчётно.

\begin{Proposition}
$\mathbb{Z}$ -- счётно.
\end{Proposition}
\begin{proof}
Определим биективное отображение из $\mathbb{N}$ в $\mathbb{Z}$ следующим образом: 
\begin{equation*}
f(x) = 
 \begin{cases}
   \frac{n}{2}, &\text{ $n = 0(mod \; 2) $}\\
   \frac{n-1}{2}, &\text{ $n = 1(mod \; 2)$}
 \end{cases}
\end{equation*}
\end{proof}

\begin{Proposition}
$\mathbb{N^2}\sim\mathbb{N}$ 
\end{Proposition}
\begin{proof}
Ну, действительно, запишем сначала все пары натуральных чисел с первым элементом равным $1$, ниже все пары, начинающиеся с $2$ и так далее. Начнём нумеровать пары следующим образом:
$$(1,1) - 1 \;\; (1,2) - 2 \;\; (1,3) - 6 \ldots$$
$$(2,1) - 3 \;\; (2,2) - 5 \;\; (2,3) - 7 \ldots$$
$$(3,1) - 4 \;\; (3,2) - 8 \;\; (3,3) - 12 \ldots$$
$$\vdots \quad\quad\quad\quad\vdots \quad\quad\quad\quad\vdots$$
Так как каждый раз диагонали будут конечными, то удастся занумеровать все пары.
\end{proof}

\begin{Proposition}
$\mathbb{Q}$ счётно.
\end{Proposition}
\begin{proof}
Очевидно, что $\mid\mathbb{Q}\mid \geq\mid\mathbb{N}\mid.$ Докажем равенство.\\
Докажем, что $\mid\mathbb{Q}\mid \leq\mid\mathbb{N}\mid:$ \\
Любое $x\in\mathbb{Q}$ представляется в виде несократимой дроби $\frac{m}{n}, \; m\in\mathbb{Z}, \; n\in\mathbb{N}.$ Тогда запишем $card\mathbb{Q}\leq card\mathbb{Z}\times\mathbb{N} = card\mathbb{N}\times\mathbb{N}=card\mathbb{N}.$\\
Отсюда имеем, что $\mid\mathbb{Q}\mid \geq\mid\mathbb{N}\mid$ и $\mid\mathbb{Q}\mid \leq\mid\mathbb{N}\mid,$ следовательно $\mid\mathbb{Q}\mid =\mid\mathbb{N}\mid$
\end{proof}

\subsubsection{Мощность континуум}
\begin{Proposition}
Множество вещественных чисел $\mathbb{R}$ и $\mathcal{P}(\mathbb{N})$ равномощны.
\end{Proposition}
\begin{proof}
\begin{enumerate}
\item $x\in\mathbb{R} \longmapsto (-\infty, x)\cap\mathbb{Q} \subset \mathbb{Q}.$ Тем самым определено инъективное отображение множества $\mathbb{R} \rightarrow\mathcal{P}(\mathbb{Q}),$ поэтому $card\mathbb{R}\leq card\mathcal{P}(\mathbb{Q})=card\mathbb{N}.$
\item Рассмотрим некое подмножество натуральных чисел $A\subset\mathbb{N}.$ Построим бесконечные десятичные дроби $x=0.\alpha_1\alpha_2\cdots$ по следующему правилу:
\begin{equation*}
\alpha_n = 
 \begin{cases}
   1, &\text{ $n\in A $}\\
   0, &\text{ $n\notin A$}
 \end{cases}
\end{equation*}
Таким образом определена индукция $\mathcal{P}(\mathbb{N}) \rightarrow [0, 1),$ 
\end{enumerate} 
значит $$card\mathcal{P}(\mathbb{N})\leq card\mathbb{R} \Rightarrow card\mathbb{R}= card\mathcal{P}(\mathbb{N}).$$
\end{proof}

\begin{Definition}
Если $X\sim\mathcal{P}(\mathbb{N})\sim\mathbb{R},$ то говорят, что $x$ имеет мощность континуум и обозначают $cardX=c=2^{\aleph_0}.$
\end{Definition}

\newpage

\begin{center}
\section{\LARGE{\bf Предел последовательности}}
\end{center}
\epigraph{\textit{Предел один, а у нас перерыв.}}
{-- Моисеев А.А.}


\begin{Definition}
Числовая последовательность это отображение $\mathbb{N} \rightarrow \mathbb{R}.$ Для значения этого отображения (последовательности) можно использовать запись $x_n;$ сама последовательность обозначается, как $\{x_n\}_{n=1}^{\infty}.$
\end{Definition} 

Пусть $\{x_n\}_{n=1}^{\infty}$ -- последовательность, тогда множеством значений последовательности будет $X=\{x\in\mathbb{R}\mid\exists n\in\mathbb{N} \; x=x_n\}.$

Последовательность называется ограниченной сверху (снизу), если соответсвующим свойством обладает её множество значений: 

Последовательность $\{x_n\}_{n=1}^{\infty}$ ограничена, если $\exists m, M \; \forall n\in\mathbb{N} \; m\leq x_n\leq M$ или $\exists M: \; \forall n\in\mathbb{N} \; \mid x_n\mid<M.$

\paragraph{Пример}
\begin{enumerate}
\item $x_n=\frac{1}{n}$ --ограничена, ведь $0\leq\frac{1}{n}\leq1.$
\item $x_n=n^2$ -- ограничена снизу: $x_n\geq 0,$ но не сверху.
\end{enumerate}

\begin{Definition}[Предел последовательности]
Число $a\in\mathbb{R}$ называется пределом последовательности $\{x_n\}_{n=1}^{\infty},$ если для любого положительного числа можно подобрать такой номер $N,$ что разность всякого члена последовательности с большим номером и $a$ будет по модулю меньше этого положительного числа: $$\forall \varepsilon>0 ;\ \exists N: \; \forall n>N \; \mid x_n-a\mid<\varepsilon.$$
\end{Definition}

Последовательность, имеющая предел, называется сходящейся.
Последовательность, не имеющая предела, называется расходящейся.
Третьего не дано.

\paragraph{Пример}
$\frac{1}{n} \xrightarrow[n\rightarrow \infty]{} 0.$

Действительно, возьмём $N=\left[\frac{1}{\varepsilon}\right] +1.$ Пусть $n>N,$ тогда $n>\left[\frac{1}{n}\right], \; n>\frac{1}{\varepsilon}, \; \frac{1}{n}<\varepsilon.$

\begin{Definition}[Окрестность]
\begin{enumerate}
\item Возьмём точку $a\in\mathbb{R},$ тогда правильной $\varepsilon$ окрестностью точки $a$ называется $O_{\varepsilon}(a)=(a-\varepsilon, a+\varepsilon) = \{x\in\mathbb{R}\mid \; \mid x-a\mid<\varepsilon\}.$ Иногда можно окрестность обозначать как $V,$ дело вкуса.
\item $O\in\mathbb{R}$ над окрестность точки $a,$ если $\exists\varepsilon>0: \; O_{\varepsilon}(a)\subset O.$
\end{enumerate}
\end{Definition}

\paragraph{Пара свойств.}

Если $O_1 \subset O_2$ и $V_1$ окрестность точки $a$, то $V_2$ тоже окретсность точки $a.$

Если $O_1$ и $O_2$ окрестности точки $a,$ то $V_1\cap V_2$ тоже окрестность точки $a.$

\begin{Definition}
Последовательность $x_n \xrightarrow[n\rightarrow \infty]{} a,$ если $\forall O$ -- окрестность точки $a\;$  $\exists N, \; \forall n>N \Rightarrow x_n\in O.$
\end{Definition}

\subparagraph{Простейшие свойста пределов}
\begin{enumerate}
\item Предел стационарной последовательности. 

$\{x_n\}_{n=1}^{\infty}, \; x_n=a.$ Пусть $n=n_0+1, n_0+2,\ldots,$ тогда $x_n\rightarrow a.$ Действительно, $\forall\varepsilon \; N=n_0 \; n>N \; x_n=a \; \mid x_n-a\mid=0<\varepsilon.$
\item Единственность предела:  $x_n\rightarrow a$ и $x_n\rightarrow b \; \Rightarrow a=b.$
\begin{proof}
Пусть $a\neq b,$ тогда рассмотрим $O_a$ и $O_b$ -- окрестности точек $a$ и $b.$ Возьмём точку $c=\frac{a+b}{2}$ и окрестности $O_a = (-\infty, c)$ и $O_b = (c, \infty).$ Тогда, так как $a$ и $b$ пределы, найдутся такие $N_1, N_2,$ что при любом $n>N_1$ и $n>N_2$ член последовательности $x_n$ будет принадлежать одновременно непересекающимся множествам $O_a$ и $O_b.$ Противоречие.
\end{proof}
\item Ограниченность сходящейся последовательности. Если $\{x_n\}_{n=1}^{\infty}$ сходится, то $\{x_n\}_{n=1}^{\infty}$ ограничена.
\begin{proof}
Пусть $x_n \xrightarrow[n\rightarrow\infty]{} a.$ Возьмём $\varepsilon=1,$ тогда существует $n_0: \; \mid x_n-a\mid<1,$ при $n=n_0+1, n_0+2,\ldots$

При таких $n\; \mid x_n\mid\leq\mid a\mid+1.$ Возьмём число $M=\max\{\mid x_1\mid, \mid x_2\mid,\ldots,\mid x_0\mid, \mid a\mid+1\}.$

Теперь $\forall n \; \mid x_n\mid<M$

если $n\leq n_0,$ то $\mid x_n\mid\leq M,$ так как $M$ максимум таких чисел.

если $n>n_0,$ то $\mid x_n\mid<\mid a\mid+1\leq M.$
\end{proof}

Однако может случиться так, что последовательность ограничена и при этом расходится: $x_n=(-1)^{n}.$

\item Подпоследовательность.

\begin{Definition}
Пусть $\{x_n\}_{n=1}^{\infty}$ -- числовая последовательность. Рассмотрим $\{n_k\}_{k=1}^{\infty}$ -- строго возрастающая последовательность натуральных чисел. Построим последовательность $\{y_k\}_{k=1}^{\infty}: \; y_k=x_{n_k}, k=1, 2,\ldots$ Тогда $\{y_k\}$ -- подпоследовательность $\{y_n\}.$
\end{Definition}
\end{enumerate}

\subparagraph{Пример}
Имеется последовательность $\{x_n\}_{n=1}^{\infty}.$
\begin{enumerate}
\item Возьмём $n_k=n_0+k$ и построим подпоследовательность $y_k=x_{n_0+k}.$ Таким образом мы отбросили первые $n_0$ элементов последовательности.
\item $n_k=x_{2k}$ -- только чётные элементы.

$n_k=x_{2k-1}$ -- только нечётные элементы.
\end{enumerate}

\begin{Proposition}
Подпоследовательность $\{y_k\}$ сходящейся последовательности $\{x_n\}$ сходится к тому же пределу, что и сама последовательность.
\end{Proposition}
\begin{proof}
Докажем, что если $\{x_n\}\rightarrow a,$ то и $y_k\rightarrow a (y_k=x_{n_k}.$

Если $n_k$  строго возрастающая последовательность, то $n_k\geq k.$ Так как $\{x_n\}$ сходится, то $\forall\varepsilon>0 \; \exists N: \; \forall n>N \Rightarrow\mid x_n-a\mid<\varepsilon.$ Пусть $k>N,$ тогда $n_k\geq k>N \Rightarrow n_k>N,$ значит $\mid x_{n_k}-a\mid<\varepsilon \; x_{n_k}=y_k \Rightarrow \mid u_n-a\mid<\varepsilon,$ то есть $y_k \xrightarrow[n\rightarrow\infty]{} a.$
\end{proof}

\begin{Remark}
Расходящаяся последовательность может иметь сходящуюся подпоследовательность:

$x_n=(-1)^k: \; x_{2k}=1$ и $x_{2k-1}=-1.$
\end{Remark}

\subsection{Сходимость и арифметические операции}
\begin{Theorem}[Предел суммы]

Пусть имеются сходящиеся подпоследовательности $\{x_n\}$ и $\{y_n\}: \; x_n\rightarrow a$ и $y_n\rightarrow b,$ тогда последовательность $\{z_n\}=x_n+y_n$ сходится к $a+b.$
\end{Theorem}
\begin{proof}
$x_n\rightarrow a \Rightarrow \forall \frac{\varepsilon}{2}>0 \; \exists N_1: \; \forall n>N_1 \; \mid x_n-a\mid<\frac{\varepsilon}{2}.$

$y_n\rightarrow a \Rightarrow \forall \frac{\varepsilon}{2}>0 \; \exists N_2: \; \forall n>N_2 \; \mid y_n-b\mid<\frac{\varepsilon}{2}.$

Возьмём $n>\max(N_1, N_2),$ сложим эти два неравенства и воспользуемся неравенством треугольника:

$\varepsilon=\frac{\varepsilon}{2}+\frac{\varepsilon}{2}>\mid x_n-a\mid+\mid y_n-b\mid \geq \mid x_n+y_n-(a+b)\mid = \mid z_n-(a+b),$ откуда $z_n \rightarrow a+b.$
\end{proof}

\begin{Theorem}[Предел произведения]
Пусть есть две последовательности $\{x_n\}$ и $\{y_n\},$  рассмотрим последовательность $\{z_n\}: \; z_n=x_n\cdot y_n.$ Тогда, если $x_n\rightarrow a, \; y_n\rightarrow b,$ то последовательность $\{z_n\}$ сходится и сходится к $z_n\rightarrow a+b.$
\end{Theorem}
\begin{proof}
Предварительная оценка: $\mid z_n-ab\mid=\mid x_n y_n-ab\mid=\mid(x_ny_n-ay_n)+(ay_n-ab)\mid\leq\mid y_n\mid\mid x_n-a\mid+\mid a\mid\mid y_n-b\mid.$ Так как $\{y_n\}$ -- сходится, то $\exists M: \; \mid y_n\mid<M \; \forall n\in\mathbb{N},$ более того, можно считать, что $\mid a\mid<M.$

Так как $x_n\rightarrow a,$ то $\forall\varepsilon\;\exists N_1: \; \forall n>N_1 \Rightarrow\mid x_n-a\mid<\frac{\varepsilon}{2M}.$

Из того, что $y_n\rightarrow b$ следует, что $\exists N_2: \; \forall n>N_2 \Rightarrow \mid y_n-b\mid<\frac{\varepsilon}{2M}.$

Возьмём $N=\max(N_1, N_2),$ таким образом, если $n>N,$ то будут выполнены оба неравенства одновременно.

Пусть $n>N,$ тогда $\mid z_n-ab\mid\leq\mid x_n=a\mid\mid y_n\mid+\mid a\mid\mid y_n-b\mid<\frac{\varepsilon}{2M}M+\frac{\varepsilon}{2M}M=\frac{\varepsilon}{2}+\frac{\varepsilon}{2}=\varepsilon \Rightarrow \mid z_n-ab\mid<\varepsilon,$ то есть $z_n\xrightarrow[n\rightarrow\infty]{} ab.$
\end{proof}

\begin{Theorem}[Предел частного]
Пусть имеются сходящиеся подпоследовательности $\{x_n\}$ и $\{y_n\}: \; x_n\rightarrow a$ и $y_n\rightarrow b,$ тогда последовательность $\{z_n\}=\frac{x_n}{y_n}, y_n\neq 0, n\in\mathbb{N}$ сходится к $\frac{a}{b}.$
\end{Theorem}
\begin{proof}
%!!!!!!!!!!!!!!!!!!!!!!!!!!!!!!!!!!!!!!!!!!!!!!!!!!!!!!!!!!!!!!!!!!!!!!!!!!!!!!!!!!!!!!!
\end{proof}
\begin{Corollary}
Если $x_n\rightarrow a, y_n\rightarrow b,$ то $z_n=\lambda x_n+\mu y_n$ сходится и $z_n\rightarrow\lambda a+\mu b.$
\end{Corollary}
\subsection{Сходимость и неравенства}
\begin{Theorem}
Если $x_n\rightarrow a, y_n\rightarrow b, a<b,$ тогда $\exists n_0: \; x_n<y_n, n>n_0.$
\end{Theorem}
\begin{proof}

\end{proof}
\begin{Corollary}[О стабилизации знака]
\end{Corollary}
\begin{Theorem}[Предельный переход в неравенстве]
Пусть $x_n\leq y_n, n\in\mathbb{N}$ и\\ $x_n\xrightarrow[n\rightarrow\infty]{} a,\; y_n\xrightarrow[n\rightarrow\infty]{} b\;  \Rightarrow \;a\leq b.$
\end{Theorem}
\begin{proof}

\end{proof}

\begin{Remark}

\end{Remark}

\begin{Remark}

\end{Remark}

\begin{Theorem}[Теорема о двух милиционерах]

\end{Theorem}
\begin{proof}

\end{proof}

\subsection{Бесконечно малые последовательности}
\begin{Definition}
Последовательность $\{\alpha_n\}$ называется бесконечно малой, если $\alpha_n \rightarrow 0.$ То есть $\exists N: \; \forall n>N \mid\alpha_n\mid<\varepsilon.$
\end{Definition}

\begin{Theorem}
Сумма бесконечно малых последовательностей является бесконечно малая последовательность.
\end{Theorem}
\begin{proof}
Смотри %ЗАПИЛИТЬ ССЫЛКУ НА ТЕОРЕМУ 2.1!!!!!!!!!!!!!!!!!!!!!!!!!!!!!!!!!!!!!!!!!!!!!!!!!!!
\end{proof}

\begin{Theorem}
Произведение бесконечно малой и ограниченной -- бесконечно малая.
\end{Theorem}
\begin{proof}
Пусть $\{\alpha_n\}$ -- бесконечно малая и $\{x_n\}$ ограниченная. $\{x_n\}$ ограничена, значит $\exists M: \; \mid x_n\mid\leq M, \forall n\in N.$ Из бесконечной малости $\{\alpha_n\}$ следует, что $\forall\varepsilon>0 \; \exists N: \; \forall n>N \; \mid\alpha_n\mid<\frac{\varepsilon}{M},$ то есть $\mid\omega_n\mid=\mid\alpha_n\mid\mid x_n\mid\leq\mid\alpha_n\mid\cdot M<\frac{\varepsilon}{M}\cdot M=\varepsilon \Rightarrow \omega_n\xrightarrow[n\rightarrow\infty]{} 0.$
\end{proof}
\subsubsection{Описание сходимости на языке бесконечно малых}
$x_n \xrightarrow[n\rightarrow\infty]{} a,  \Leftrightarrow \alpha_n=x_n-a$ -- бесконечно малая.

\subsection{Бесконечно большие последовательности}

\subsection{Расширенная числовая прямая}
$\overline{\mathbb{R}}=\mathbb{R}\cup\{-\infty, +\infty\}$ -- Риманова прямая.

$\forall x\in\mathbb{R} \; -\infty<x<+\infty$

Над $\pm\infty$ выполняют ряд операций (с некоторыми ограничениями).

\subsection{Сходимость монотонной последовательности}

\subsection{Число $e$}
Рассмотрим последовательность $x_n=\left(1+\frac{1}{n}\right)^n.$
\begin{eqnarray}
\nonumber x_n=1+C_n^1\frac{1}{n}+C_n^2\frac{1}{n^2}+\ldots+C_n^n\frac{1}{n^n} = \qquad\qquad\qquad\qquad\qquad\qquad\qquad\qquad\;\;\;\quad\quad\\ \nonumber
=1+n\frac{1}{n}+\frac{n(n+1)}{2!}\frac{1}{n^2}+\frac{n(n-1)(n-2)}{3!}\frac{1}{n^3}+\ldots+\frac{n(n-1)\ldots(n-n+1)1}{n!}\frac{1}{n^n}=\\ \nonumber
=1+1+\frac{1}{2!}\left(1-\frac{1}{n}\right)+\frac{1}{3!}\left(1-\frac{1}{n}\right)\left(1+\frac{2}{n}\right)+\ldots+\frac{1}{n!}\left(1-\frac{1}{n}\right)\ldots\left(1-\frac{n-1}{n}\right).
\end{eqnarray}

Таким образом $x_n=1+1+\frac{1}{2!}\left(1-\frac{1}{n}\right)+\frac{1}{3!}\left(1-\frac{1}{n}\right)\left(1+\frac{2}{n}\right)+\ldots+\frac{1}{n!}\left(1-\frac{1}{n}\right)\ldots\left(1-\frac{n-1}{n}\right)$\\


и $x_{n+1}=1+1+\frac{1}{2!}\left(1-\frac{1}{n+1}\right)+\ldots+\frac{1}{n!}\left(1-\frac{1}{n+1}\right)\ldots\left(1-\frac{n-1}{n+1}\right)+\frac{1}{(n+1)!}\left(1-\frac{1}{n+1}\right)\ldots\left(1-\frac{n}{n+1}\right).$\\

Как видно, переходе от $x_n$ к $x_{n+1}$ все слагаемые увеличиваются$: \; x_{n+1}>x_n.$

$\{x_n\}_{n=1}^{\infty}$ -- возрастает.

$x_n\leq 1+1+\frac{1}{2!}+\frac{1}{3!}+\ldots+\frac{1}{n!}\leq 1+1+\frac{1}{2}+\frac{1}{2^2}+\ldots+\frac{1}{2^{n-1}}\leq 1+2=3.$

Получается, что $x_n$ ограничена сверху тройкой. Тогда по теореме %ЗАПИЛИТЬ ССЫЛКУ НА  ТЕОРЕМУ
$\{x_n\}$ -- сходится.

\begin{Definition}
Предел этой последовательности обозначается числом $"e".$

$e=\lim\limits_{n\rightarrow\infty}\left(1+\frac{1}{n}\right)^n \qquad\qquad e\approx 2.71828182459045\ldots$
\end{Definition}

\subsection{Верхний и нижний предел последовательности}
\begin{Definition}
Пусть $\{x_n\}_{n=1}^\infty$ -- ограниченная последовательность, тогда $X_n=\{x_n, x_{n+1},\ldots\}$ -- ограниченное множество.

$y_n=\inf X_n$ -- убывает и ограничена$; \;z_n=\sup X_n.$ -- возрастает и ограничена, значит $y_n$ и $z_n$ сходятся.\\
$z_n\rightarrow L-\overline{\lim} \;x_n$ -- верхний предел\\
$y_n\rightarrow l=\underline{\lim} \;x_n$ -- нижний предел.
\end{Definition}

\begin{Theorem}
Пусть последовательность $\{x_n\}_{n=1}^\infty$ -- ограничена, тогда последовательность сходится $\Leftrightarrow \; \underline{\lim} x_n=\overline{\lim} x_n.$ В случае сходимости $\lim x_n=\underline{\lim} x_n=\overline{\lim} x_n.$
\end{Theorem}
\begin{proof}
Докажем необходимость. $\Rightarrow:$\\
Пусть $x_n\xrightarrow[n\rightarrow\infty]{} A,$ тогда $\forall\varepsilon>0\;\exists N: \; \forall n>N\Rightarrow\;\mid x_n-A\mid<\varepsilon,$ иными словами $A-\varepsilon\leq y_n\leq A-\varepsilon\Rightarrow y_n\rightarrow A\leftarrow z_n \; \Rightarrow \; A=\overline{\lim} x_n =\underline{\lim} x_n.$

Докажем достаточность. $\Leftarrow:$

$A=\underline{\lim}x_n=\overline{\lim}x_n.$ Имеет место неравенство$: \; $%КОММУТАТИВНАЯ ДИАГРАММА!!!!!!!!!!!!!!!!!!!!!!!!
\end{proof}
\subsection{Принцип выбора Больцано-Вейерштрасса}
\begin{Theorem}
Из ограниченной последовательности можно извлечь сходящуюся подпоследовательность. Если $x_n$ -- ограничена то $\exists\{n_k\}$ -- строго монотонно возрастающая, такая, что $\{x_{n_k}\}$ сходится.
\end{Theorem}
\begin{proof}
Построим подпоследовательность, сходящуюся к $L=\overline{\lim\limits_n}x_n.$

$X_n=\{x_n, x_{n+1},\ldots\}.$ Так как последовательность ограничена, то будет существовать $z_n=\sup X_n$ и $z_n\rightarrow L$
\end{proof}
\subsection{Критерий Коши}
\newpage

\begin{center}
\section{\LARGE{\bf Предел функции}}
\end{center}
\epigraph{\textit{Ну а это вы изучите сами дома.}}
{-- Моисеев А.А.}
\subsection{Понятие предела функции}
\begin{Definition}
Пусть $E\subseteq\mathbb{R}$ и $a\in\mathbb{R}.$ $a$ называется предельной точкой для $E,$ если $\forall\delta>0\;\exists x\in E: \; 0<\mid x-a\mid<\delta.$
\end{Definition}

Рассмотрим множество на числовой прямой $(a, b)\cup\{c\}.$ Точки $a$ и $b$ -- предельные точки множества. Точка $c$ называется изолированной точкой множества.

%КАРТИНКА!!!!!!!!!!!!!!!!!!!!!!!!!!!!!!!!!!!!!!!!!!!!!!!!!!!!!!!!!!!!!!!!!!!!!!!!

$a$ предельная, если $\forall O$ окрестности $a \; \exists x\in O\cap E, x\neq a.$

Помимо обычной окрестности точки вводят также определение проколотой окрестности: $\dot{O} = O\setminus\{a\}.$

Таким образом, $a$ предельная точка для $E,$ если $\forall O$ -- окрестности точки $a$ выполнено, что $E\cap\dot{O}\neq\varnothing.$

\subsection{Определение предела функции}

\begin{Definition}
$E\subseteq\mathbb{R}, a$ -- предельная точка $E, \; f$ -- функция на $\mathbb{E}, A\in\mathbb{R}. \; A$ называется пределом функции $f$ в точке $a, \;f(x)\xrightarrow[x\rightarrow a]{} A; \; A=\lim\limits_{x\rightarrow a} f(x); A=\lim\limits_{a}f,$ если:
\begin{enumerate}
\item на языке $\varepsilon$-$\delta:$

$\forall\varepsilon>0 \;\exists\delta>0\;\forall x\in E 0<\mid x-a\mid<\delta \Rightarrow \mid f(x)-A\mid<\varepsilon;$

\item на языке окреcтностей$:$

$\forall O$ окрестности точки $A$ $\exists V$ -- окрестность точки $a: \; f(\dot{V}\cap E \subset O;$

\item на языке последовательностей$:$

$\forall\{x_n\} \; x_n\in E \; x_n\neq a, \; x_n\xrightarrow[n\rightarrow\infty]{} a,  \Rightarrow f(x_n)\xrightarrow[n\rightarrow\infty]{} A.$
\end{enumerate}
\end{Definition}

\subparagraph{Пример}

%КАРТИНКА!!!!!!!!!!!!!!!!!!!!!!!!!!!!!!!!!!!!!!!!!!!!!!!!!!!!!!!!!!!!!!!!!!!!!!!!

Предел -- локальное понятие. Если $f=g$ в $\dot{O}$ проколотой окрестности точки $a,$ то они имеют пределы и они равны.

\end{document} 
